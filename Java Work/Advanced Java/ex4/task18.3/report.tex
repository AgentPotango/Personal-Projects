\documentclass[12pt,a4paper]{article}
\usepackage{url}
\usepackage[parfill]{parskip}
\title{Final Report}
\author{Kashif Hussain}
\date{9/05/2013}
\begin{document}
\maketitle
\tableofcontents

\newpage

\section{Preface}
This document will cover my experience of working with a group.
The main topics will cover:

\begin{itemize}

\item  What tools and techniques have I have learned about?
\item  What concepts and principles I have learned about?
\item  What skills I have developed?
\item  How effectively did my team and I function?
\item  The Future.

\end{itemize}

\newpage
\section{Tools and techniques have I have learned about}
I have learnt about many tools and techniques over the course; some to do with coding 
and others to do with group work and personal learning.

A list of Tools I have learnt:
	
\textbf{PHP}, \textbf{MySQL}, \textbf{HTML}, \textbf{CSS}, \textbf{JavaScript},
\textbf{Ajax}.

\textbf{Online Coding Validators}\cite{W3 Validator} and \textbf{Learning To Code}\cite{W3 Schools}

A list of Techniques:

\textbf{Time Management}, 
\textbf{Tracking TeamWork} and  \textbf{Presenting Skills}.

\subsection{What do they do?}
The languages have helped us in creating the website and making it more interactive.
Each language or tool had their own special use which are detailed below.

\textbf{PHP:}   
		\\*\\*PHP is used to perform calculations like figuring out the day of the week.
                It is also used to interact with MySQL databases and allows you to create dynamic web pages
                such as a customized user page.

\textbf{MySQL:} 
		\\*\\*MySQL Is a database used to store information which you can used to create
		dynamic web pages.
		
\textbf{HTML:}  
		\\*\\*HTML elements form the building blocks of all websites and is used to construct
		the most basic web page. It is used to create pages of text, images, interactive forms
		and more.
\newpage	
\textbf{CSS:}   
		\\*\\*CSS is a style language that defines layout of HTML documents. 
		For example, CSS covers fonts, colours, margins, lines, height, width, 
		background images and more. CSS enables you to add a house style to your website.

\textbf{JavaScript:} 
		\\*\\*JavaScript is used for relatively simple functions such as popping up new windows. 
		It is used to add interactivity with your website and can be used to make sure
		that all the form details entered obey a given set of rules.

\textbf{Ajax:}  
		\\*\\*Ajax is used for creating interactive web applications. It makes web pages feel more responsive 
		and allows information to be loaded behind the scenes so that the entire web page does not have to 
		be reloaded each time the user requests a change. 
		It increase the web page's interactivity, speed, and usability.
		
\textbf{Online Coding Validators:} 
		\\*\\*These websites check your code for any errors or incompatibility with certain 
		web browsers. 

\textbf{Learning To Code:}
		\\*\\*These websites teach you how to code in certain languages.
		
Whilst all these tools are useful in creating a website, certain techniques
were equally important, if not more. They are the following:

\textbf{Time Management:}
		\\*\\*Time management is important to get things done in a given time frame
		and to set deadlines for pieces of work
		
\textbf{Monitoring Team Work:}
		\\*\\*Version control, facebook and email can all be used to monitor work
		being done by the team. 
		
\textbf{Presentation Skills:}
		\\*Presentation skills increases the audience's interest in what you are \\presenting. 

\newpage   
\subsection{Why are they useful or important?}

The coding languages were instrumental in making our website more dynamic and interactive.
PHP and MySQL allowed us to control the data we stored and modify it in any way we saw fit to. 
HTML and CSS formed the basic layout of our web pages and allowed us to make the website
both appealing to look at and structured so that the information you need or want to look 
at is easily visible. JavaScript and Ajax allowed us to create some simple functions which
increased the usability and speed of our website in general.

The techniques I learned about helped my team and I to manage our time and set deadlines 
to get work done. Monitoring team work was essential as it told us who was working on a specific
part of the website and so prevented us having two people working on the same page.
Presentation skills helped us present our website design and taught us to use many
visual aids and interact more with the audience.

\subsection{How did I learn about them?}

The coding languages were learnt through a combination of search engines and the
various apps on my tablet.
The techniques I learnt were mostly through experience and feedback from the people I knew.

Time management was learnt through building the website as some of our functions we wished
to implement simply weren't worth the time or effort. Through presentation rehersals,
I recieved feedback from my team members which enhanced my presentation skills.

Whilst I was working on the website, I experienced problems where my group members and I were
working on the same pages which caused conflict in our design and implementation. This taught me
that learning to monitor what each member of your team is doing is really important.

\newpage
\section{Concepts and principles I learnt}

\textbf{DRY - Don’t repeat yourself:}
		\\*\\*In regards to our website, this meant that if we changed some code related
		to our homepage, then it does not require us to change anything about our
		other web pages.
		
\textbf{Commenting is important:}
		\\*\\*Commenting increases code readability and allows you to go back to your code
		to understand what you have written and why it works.
		It also allows your team to understand your code.
		
\textbf{Clearly articulated, shared goal:}
		\\*\\*Having a shared goal which has been clearly articulated to the other team members
		is vital. This means that everyone knows what is expected of them and what they are aiming
		for and, as a result, reduced questions about how the code they are working on will be relevant
		to the website.

\textbf{Supporting and building on ideas:}
		\\*\\*Supporting and building on ideas allows you to add more depth to your project
		and gives you multiple perspectives on an idea. This ensures a collaborative
		discussion between your team members and allows you to further clarify the goals
		you want to achieve.
		
\newpage
\section{The skills I have developed}

\subsection{Skills I learnt to improve my personal learning}

Actively listening has allowed me to understand the topic I'm listening too and has 
also helped me take part in a discussion about said topic.
Clarification has helped me ease out any uncertainties I may have had regarding a Maths or Computer
Science topic.

Creating learning targets has helped me pace out my work load and has also given me a sense
of accomplishment by reaching them. Deadline setting has equally been important for me
as it reduced the stress in needing to complete a task before an actual time frame given
by the courses I'm taking.

\subsection{Skills I learned to improve my effectiveness as a team member}

Becoming flexible is one of the biggest skills I have learnt by working in a team.
Flexibility was needed when team members didn't fully understand how to code a function,
or how to manipulate the data in the MySQL databases. 

Being able to share my ideas openly and willingly is another skill I have learnt. Previously,
I used to wait for the group to ask me before sharing my idea, but that has changed over the time
I worked with them.

Monitoring my team's work has also helped me become an effective team member. 
Being able to monitor their work has helped me in setting deadlines or targets and
helping them if they were stuck on coding a certain part of the website we worked on. 

\newpage
\subsection{How are these related?}        

The skills I learnt to improve my personal learning allow me to become a more effective team member.
Actively listening has allowed me to understand  a topic better which furthers the knowledge I can 
regurgitate to my team members. Creating learning targets and deadlines for myself has familiarized
the target and deadlines teams set and so I can already work effectively in accordance to those
targets/deadlines.

Likewise, the skills i learnt to improve my effectiveness as a team member can also improve my personal
learning. Being able to share my ideas more openly and willingly allows others to discuss my ideas
and in turn, helps me see, from another perspective the problems with my idea or how my idea is actually great.
Becoming flexible also improves my personal learning as I now have the ability to engage in discussions
which cover the new set of skills I have learnt. 

Both set of skills, in general, improve your ability to learn and your ability to
translate what you have learnt into the work you are doing.

\newpage
\section{How effectively did my team and I\\ function?}

\subsection{The positive aspects of working in my team}

Our ability to discuss our ideas openly and critique them has been really helpful
in designing our website as we were able to choose what we wanted in our website
pretty quickly and so a lot of time could be spent on development

Everyone being able to learn about the how they can implement something independently
is one of the positives of working with my team. This allowed everyone to get on with 
the task they were given and allowed us to concentrate on the harder functions together. 

\subsection{The negative aspects of working in my team}

Laziness from group members has been one of the negatives of working with the team.
This lead to deadlines being missed constantly or improper time management by some.
As a result, some people did more than others.

Ineptitude in certain coding languages restricted many of our team members on what they could
do. This was a direct result of their procrastination and lead to many problems such as the fact
they could do none of the difficult tasks which needed to be done by them and instead, was placed
on others.

\newpage
\subsection{The problems we encountered and how we coped with them}

Problems in missing deadlines or inability to code certain parts of our website were solved
by mainly giving the work to someone who did know how to do it. This resulted in a lot of work
for some people and very little for others.

Ineptitude in certain languages was solved by giving those team members more work to do with 
the programming language that they know how to use.

\subsection{How we ensured everyone took part}

Those who did not know certain languages were helped by someone who did know how to use
that language.This resulted in the person inept in that coding language learning how to use 
some of it to do complete their task.

We constantly updated each other on the work we have to do through the use of facebook and
had many meetings to discuss the work everyone needed to do. We always made sure everyone understood
what was expected of them by the end of the meetings.

\newpage
\section{The Future}

\subsection{What skills I need to concentrate on improving and how this can be achieved}

Improving the readability of the code I write is one of the skills I need to improve.
Adding comments to do with my code should help fix this.

I also need to get my work more organized and this can be achieved by creating a timetable 
for the work which needs to be done and putting my work into folders which I can easily navigate through.

\subsection{How the organisation of the project course unit can be improved for future years}

I think that educating people in the use of a version control unit could help people greatly
when working on their project. Many of my team members were unable to use the one supplied by the
university and so we were unable to use it.

The course could also cover coding languages such as JavaScript and teach how to implement how some
of the most basic functions. This would help the student's understanding of how the language is coded
and how it can be used for other functions.

\subsection{Advice for a first year regarding the project}

It is important to start your design early and identify your problems early. Doing this will allow
you to concentrate more on the implementation of your website rather than discussing what you could 
add to the website or what you want to change. Having a clear design choice will pave the way for more
development time. Start too late and you will end up doing poorly. Start midway and you'll end up with regrets
about how you could have done better.

\newpage

\raggedright
\sloppy
\begin{thebibliography}{99}
\bibitem{W3 Validator}
\url{http://validator.w3.org/}

\bibitem{W3 Schools}
\url{http://www.w3schools.com/}

\end{thebibliography}
\end{document}
