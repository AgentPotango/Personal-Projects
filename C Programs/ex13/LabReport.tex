\documentclass{article}

\title{COMP26120 Lab 13: Background}
\author{?}

\begin{document}
\maketitle

% PART 1 %%%%%%%%%%%%%%%%%%%%%%%%%%%%%%%%%%%%%%%%%%%%%%%%%%%%%%%%%%%%%%%%%%%%%%

\section{The small-world hypothesis}
\label{sec:small world}
% Here give your statement of the small-world hypothesis and how you
% are going to test it.

The small-word hypothesis is that (in the context of the exercise) all the friends
from the two universities can be found by a very few number of steps.
It is linked to "6 degrees of seperation" which suggests that everyone is related 
to everyone else.

Everyone and everything is six or fewer steps away
from any other person in the world, so that a chain of 
"a friend of a friend" statements can be made to connect any two people 
in a maximum of six steps

I will test the small-world hypothesis by using the .gx files and finding out if
the distance from all the nodes (the friends) to each other is 6 or fewer steps by the use
of dijkstra's algorithm. 


\section{Complexity Arguments}
\label{sec:complexity}
% Write down the complexity of Dijkstra's algorithm and of Floyd's algorithm.
% Explain why, for these graphs, Dijkstra's algorithm is more efficient.

The complexity of Dijkstras algorith is O(EVLogV) where E is edges and V is vertices
This is because I use a binary heap implementation where as a fibonacci heap implementation
would use O(E + VLogV)

Floyd's algorithm's worst case is O(V^3) which is much worse than dijkstra's as 
it is dependent on the number of vertices where each node can have 1+ of.

As these graphs have lots of vertices and dijkstras uses heap sort which is VlogV instead of V^3,
dijkstra's is much better to use for these graphs


\section{Part 1 results}
\label{sec:part1}
% Give the results of part one experiments.

I found that not all of the nodes were 6 degrees of seperation from each other.

However, Most nodes from both Caltech and Oklahoma were < 6 degrees of 
seperation from each other which supports the small world hypothesis

% PART 2 %%%%%%%%%%%%%%%%%%%%%%%%%%%%%%%%%%%%%%%%%%%%%%%%%%%%%%%%%%%%%%%%%%%%%%

\section{Part 2 complexity analysis}
\label{sec:complexity2}
% Give the complexity of the heuristic route finder.

The worst case complexity for the heuristic route finder is:
Source  =  first node
target = last node

In the worse case the target is only in the second to last node's 
outlist and the outlist and there never is an empty outlist

Then each node has a set amount of vertices
For each vertex, find the outdegree -> (N * V)


-> (N*(V + 1 + 1 + 1)) = N*V

So no, in its worst case of going through every node, it is N*V which
is faster than N*VLogV. if the fibonnacci heap was used, then dijkstra's is
faster.


\section{Part 2 results}
\label{sec:part2}
% Give the results of part two experiments.

The path to finding the node with the largest outdegree was very efficient
Instead of going through all the nodes like dijkstra's the algorithm just
took the next node with the largest outdegree


\end{document}

